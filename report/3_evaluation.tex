\section{Evaluation}
\label{sec:evaluation}
%Evaluation:
%	Purpose of the evaluation
%	Data used in the evaluation
%	Evaluation setup
%	Results (present as many results as necessary to illustrate the work you have done on the project - tables are a good way and required by the assignment; plots can be another good way)

% purpose of evaluation
\subsection{Evaluation Criteria}
In order to compare performances of our systems we used the ...

% data used in the evaluation
\subsection{Dataset}

The dataset used are ...

\begin{loglisting}
\begin{algorithmic}
\State 0 qid:18219 1:0.052893 2:1.000000 3:0.750000 4:1.000000 ... 46:0.966667
\State \hspace*{25pt} \#docid = GX004-93-7097963 inc = 0.0428115405134536 prob = 0.860366
\State 1 qid:18219 1:0.026446 2:0.750000 3:0.750000 4:0.500000 ... 46:0.266667
\State \hspace*{25pt} \#docid = GX020-25-8391882 inc = 1 prob = 0.115043
\State 0 qid:18219 1:0.029752 2:0.000000 3:1.000000 4:1.000000 ... 46:0.100000
\State \hspace*{25pt} \#docid = GX025-94-0531672 inc = 1 prob = 0.141903
\end{algorithmic}
\caption{Example rows from XXXX}
\label{alg:dataset_example}
\end{loglisting}


\subsection{Evaluation setup}
The experiment was run ...

\subsection{Results}
In Table~\ref{table:results} one can see the results of the experiments.
It can be seen that ...

\begin{table}[h]
\centering
\begin{tabular}{l|lll|}
\cline{2-4}
                                          & Log-likelihood  & Perplexity        & Computation Time \\
\cline{1-4}
\multicolumn{1}{|l|}{Click Model A}       & 0               & 0                 &   \\
\multicolumn{1}{|l|}{Click Model B}       & 0               & 0                 &   \\
\cline{1-4}
\end{tabular}
\caption{Results}
\label{table:results}
\end{table}
