\section{Click Models}
\label{sec:methodology}
In this section, we briefly describe their main characteristics and differences. We have implemented CTR, PBM, CM, CCM, and TCM ourselves, the UBM algorithm is taken from PyClick \cite{PyClick}. The general notation used in this report will be as follows:

\begin{table}[ht]
	\centering
	\begin{tabular}{l|lll|}
		\hline
		Symbol & Description \\
		\hline
		$u$	& Document $u$ \\
		$q$	& Query $q$ \\
		$ S $ & Satisfaction variable \\
		$ \sigma $ & Satisfaction parameter \\
		$ E $ & Examination variable \\
		$ \epsilon $ & Examination parameter \\
		$ R $ & Relevance variable \\
		$ r $ & Relevance parameter \\
		$ C $ & Click variable \\
		$ c $ & Actual click \\
		$ \mathcal{S} $ & All sessions \\
		$ s $ & A query session \\
		$ j $ & Position within a query session \\
		\hline
	\end{tabular}
	\caption{Notations used in the equations}
	\label{table:notations}
\end{table}

\subsection{Click-through Rate click model}
The most simple click model, Click-Through Rate (CTR), tries to predict relevance for a query-document pair. Relevance is the only parameter in this model which is the click-through rate formula. The inference is done by using maximum likelihood estimation (MLE) because there are no latent variables in this model. Therefore, the click probability of document at position $u$ will be:
\begin{align}
	P(C_{uq}=1) &= P(R_{uq}=1) \nonumber \\
	P(R_{u}=1) &= \frac{1}{|S_{uq}|} \sum_{s \in S_{uq}} c_{uq} \label{eq:ctr_rel}\\
	\text{where}&, S_{uq} = \{ s_q : u \in s_q \} \nonumber
\end{align}

\subsection{Position based model}
This model builds upon the CTR model. It adds a \texttt{position bias} where documents in a higher position are examined more often. A document can only be clicked if it was examined. The click probability of a document $u$ is as is defined in Equation~\ref{eq:PBM_click} where $P(R_{uq}=1)$ is defined as in Equation~\ref{eq:ctr_rel}. The values of $P(E_r=1)$ will be fixed. They are taken from \cite{eye_track} and are $[.68, .61, .48, .34, .28, .2, .11, .1, .08, .06]$. Inference of this model is done by using expectation-maximization (EM)
\begin{align}
	P(C_{uq}=1) &= P(E_{j_u}=1)P(R_{uq}=1) \label{eq:PBM_click}
\end{align}

where $j_u$ is the rank of document $u$.

\subsection{Cascade model}
Cascade model (CM) is another extension to the CTR model. It assumes that users abandon a search session after the first click and hence does not provide a complete picture of how multiple clicks arise in a query session and how to estimate document relevance from such data \cite{Kempe2008,Craswell2008}. This model introduces the \textit{cascade hypothesis}, a used examines a search result page (SERP) from top to bottom, deciding whether to click each result before moving to the next. This means ofcourse that to make a click, the user must have decided both to click and skip the ranks above. The relevance of a document is calculated in the same way as in Equation~\ref{eq:ctr_rel}. The click probability is defined as:
\begin{align}
	P(C_{uq}=1) &= P(E_{j_u}=1)P(R_{uq}=1) \label{eq:CM_click} \\
	P(E_{j_u}=1) &= \prod_{i=1}^{j_u-1} 1 - P(C_{i_u}=1) \nonumber
\end{align}
The inference of the parameters of CM is done using MLE-inference.

\subsection{User Browsing Model}
In \cite{Dupret2008}, Dupret and Piwowarski propose a new click model called the User Browsing Model (UBM). The main difference between UBM and other models is that UBM takes the distance into account from the current document \(u_j\) to the last clicked document \(u_{j'}\) for determining the probability that the user continues browsing:
\[P(E_{j_u} = 1 \mid C_{u_{j'}}=1, C_{u_{j'+1}}=0, \dots, C_{u_{j-1}q}=0) = \gamma_{jj'}\]

A graphical representation of the model is presented in Figure~\ref{fig:ubm_gm}.

\subsection{Click Chain Model}
In \cite{Guo2009_CCM}, Fan Guo et al, proposed a bayesian based click model which was based on position bias assumption, named Click chain model (CCM). 
CCM has the following assumptions, some of which are shared with the cascade model.
\begin{enumerate}
	\item Users are homogeneous: their information needs are similar given the same query; 
	\item Decoupled examination and click events: click probability is solely determined by the examination probability and the document relevance at a given position; 
	\item Cascade hypothesis: examination is in strictly sequential order with no breaks.
\end{enumerate}

The intuition behind CCM is that the chance that a user continues after a click depends on the relevance of the previous document and that a user might abandon the search after a while. This model can be formalized with the following conditional probabilities:
\begin{align}
	P(C_{uq}=1|E_{j_u}=0) &= 0 \nonumber\\
	P(C_{uq}=1|E_{j_u}=1, R_{uq}) &= R_{uq} \nonumber\\
	P(E_{j_u+1}=1|E_{j_u}=0) &= 0 \nonumber\\
	P(E_{j_u+1}=1|E_{j_u}=1,C_{uq}=0) &= \tau_1 \nonumber\\
	P(E_{j_u+1}=1|E_{j_u}=1,C_{uq}=1,R_{uq}) &= \tau_2(1-R_{uq})+\tau_3 R_{uq} \label{eq:ccm_simplified}
\end{align}

As $\tau_2$ and $\tau_3$ are only used in Equation~\ref{eq:ccm_simplified} and in conjunction with each other and the relevance of the previous document we have decided to combine them into one parameter, $\tau_{click}$; and $\tau_1$ has been renamed to $\tau_{no\_click}$. A graphical representation of the model is presented in Figure~\ref{fig:ccm_gm}.

\begin{figure}[ht]
	\begin{subfigure}[b]{.45\textwidth}
		\centering
		\begin{tikzpicture}
			% Define nodes
			\node[obs, minimum size=1cm]                      				(c) {$C_{uq}$};
			\node[latent, left=.9cm of c, minimum size=1cm]  	(a) {$R_{uq}$};
			
			\node[latent, left=.9cm of c, yshift=2cm, minimum size=1cm]  	(e) {$E_{j_u}$};	
			\node[const, left=1cm of a]  									(a_u) {$r_u$};
			
			\node[const, left=1cm of e, minimum size=1cm]  	(e_p) {$\gamma_{jj'}$};	
			
			% Connect the nodes
			\edge {a,e} {c} ; %
			\edge {e_p} {e} ; %
			\edge {a_u} {a} ; %
			
			% Plates
			\plate [inner sep=.5cm, text centered] {u_j} {(a)(e)(c)} {document $u_j$};
			
		\end{tikzpicture}
		\caption{UBM.}	
		\label{fig:ubm_gm}
	\end{subfigure}
	\hfill
	\begin{subfigure}[b]{.45\textwidth}
		\centering
		\begin{tikzpicture}
			
			% Define nodes
			\node[obs, minimum size=1cm]                      	(c) {$C_{uq}$};
			\node[latent, left=.9cm of c, minimum size=1cm]  	(a) {$R_{uq}$};			
			\node[latent, left=.9cm of c, yshift=2cm, minimum size=1cm]  	(e) {$E_j$};	
			
			\node[obs, minimum size=1cm, right=3cm of c]        (c_1) {$C_{u_{j+1}}$};
			\node[latent, left=.9cm of c_1, minimum size=1cm]  (a_1) {$R_{u_{j+1}}$};		
			\node[latent, left=.9cm of c_1, yshift=2cm, minimum size=1cm] 	(e_1) {$E_{j+1}$};	
			
			%\node[const, above=.8cm of a]  									(a_u) {$\alpha_u$};
			%\node[const, left=1cm of e, minimum size=1cm]  	(e_p) {$\gamma_{jj'}$};	
			
			% Connect the nodes
			\edge {a,e} {c} ; %
			\edge {a,e,c} {e_1} ; %
			%\edge {e_p} {e} ; %
			%\edge {a_u} {a} ; %
			\edge {a_1,e_1} {c_1} ; %
			
			% Plates
			\plate [inner sep=.5cm, text centered] {u_j} {(a)(e)(c)} {document $u_j$};
			\plate [inner sep=.5cm, text centered] {u_j_1} {(a_1)(e_1)(c_1)} {document $u_{j+1}$};
			
		\end{tikzpicture}
		\caption{CCM.}
		\label{fig:ccm_gm}
	\end{subfigure}	
	\caption{The graphical model of UBM and CCM.}
\end{figure}

\subsection{Task-centric Click Model}
\label{sec:methodology_tcm}
The Task-centric Click Model (TCM) was first proposed by Zhang et al. in \cite{Zhang2011}. In the paper they propose a new click model which can handle multiple clicks of multiple queries in a task by introducing two new biases. The first bias indicates that users tend to express their information needs incrementally in a task, thus perform more clicks as their needs become clearer. The other bias indicates that users tend to click fresh documents that are not included in the results of previous queries. In their paper, they named the first assumption as \texttt{query bias}, and the second assumption as \texttt{duplicate bias}. A graphical representation of the state-of-the-art of the model is presented in Figure~\ref{fig:tcm_gm} and the notations used in TCM are described in Table~\ref{table:tcm_notations}. 

\begin{table}[ht]
	\centering
	\begin{tabular}{l|lll|}
		\hline
		Symbol & Description \\
		\hline
		$M_q$			& Whether the query $q$ matches the user's intent.\\
		$N_q$ 			& Whether the the user submits another query after $q$ session.\\		
		$H_{uq}$ 		& Previous Examination of the document $u$ in query $q$.\\
		$F_{uq}$ 		& Freshness of the document $u$ in query $q$.\\
		$u'q'$ & $q'$ is the latest query session where \(u\) has appeared in previous query sessions \\
		\hline
	\end{tabular}
	\caption{Additional notations used in TCM}
	\label{table:tcm_notations}
\end{table}

This model can be formalized with the following conditional probabilities:
\begin{align}
	P(M_q=1) &= \tau_1 \\
	\label{eq:alpha_2}
	P(N_q|M_q=1) &= \tau_2 \\
	P(F_{uq}=1|M_{uq}=1) &= \tau_3 \\
	P(E_{uq}=1) &= \epsilon_j \\
	P(R_{uq}=1) &= r_u \\
	M_q = 0 &\Rightarrow N_i = 1\\
	H_{uq} = 0 &\Rightarrow F_{uq} = 1\\
	H_{uq} = 0 &\Leftrightarrow H_{u',q'} = 0, E_{u',q'} = 0\\
	C_{uq} = 1 &\Leftrightarrow M_q = 1, E_{uq} = 1, R_{uq} = 1, F_{uq} = 1
\end{align}

In our implementation, we simplified TCM model by assuming that a query matches a users intent ($M_q$) is observed from the click log data, if a user clicks in a result in a query then that query matched the users intent. thus eq.\ref{eq:alpha_2} can be removed. 
Our second assumption is that $M_q, E_{uq},R_{u}$ and $F_{uq}$ are independent. 
The detailed calculation for updating EM parameters of the simplified TCM can be found in the appendix.
The graphical model of our TCM implementation is presented in Figure~\ref{fig:tcm_gm_new}.

\begin{figure}[ht]
	\begin{subfigure}[b]{.45\textwidth}
		\centering
		\begin{tikzpicture}
		
		% Define nodes
		\node[obs]                      (n) {$N_q$};
		\node[latent, left=1.2cm of n]  (m) {$M_q$};
		\node[obs, below=0.6cm of n]    (c) {$C_{uq}$};
		\node[latent, left=1.2cm of c]  (e) {$E_{uq}$};
		\node[latent, right=1.2cm of c] (r) {$R_{uq}$};
		\node[latent, below=0.6cm of c] (f) {$F_{uq}$};
		\node[latent, left=1.2cm of f]  (h) {$H_{uq}$};
		\node[latent, left=1.2cm of h]  (h_prime) {$H_{uq}'$};
		\node[latent, left=1.2cm of h, yshift=1.2cm]  (e_prime) {$E_{uq}'$};
		
		% Connect the nodes
		\edge {m} {n,c} ; %
		\edge {e,r,f} {c} ; %
		\edge {h} {c} ; %
		\edge {h_prime,e_prime} {h} ; %
		
		% Plates
		\plate [inner sep=.3cm] {p_n} {(e)(c)(r)(f)(h)} {$j=1,\dots,n$} ;
		\plate [inner sep=.3cm] {p_m} {(m)(n)(p_n)} {$i=1,\dots,m$} ;
		
		\end{tikzpicture}
		\caption{State-of-the-art TCM}
		\label{fig:tcm_gm}
	\end{subfigure}
	\hfill
	\begin{subfigure}[b]{.45\textwidth}
		\centering
		\begin{tikzpicture}
	
		% Define nodes
		\node[obs]					  (c) {$C_{uq}$};
		\node[obs, left=1.2cm of n, yshift=1.4cm]  (m) {$M_i$};
		\node[latent, left=1.2cm of c]  (e) {$E_{uq}$};
		\node[latent, right=1.2cm of c] (r) {$R_{uq}$};
		\node[latent, below=0.6cm of c] (f) {$F_{uq}$};
		\node[latent, left=1.2cm of f]  (h) {$H_{uq}$};
		\node[latent, left=1.2cm of h]  (h_prime) {$H_{uq}'$};
		\node[latent, left=1.2cm of h, yshift=1.2cm]  (e_prime) {$E_{uq}'$};
		
		% Connect the nodes
		\edge {m} {n,c} ; %
		\edge {e,r,f} {c} ; %
		\edge {h} {c} ; %
		\edge {h_prime,e_prime} {h} ; %
		
		% Plates
		\plate [inner sep=.3cm] {p_n} {(e)(c)(r)(f)(h)} {$j=1,\dots,n$} ;
		\plate [inner sep=.3cm] {p_m} {(m)(n)(p_n)} {$i=1,\dots,m$} ;
		
		\end{tikzpicture}
		
		\caption{Simplified TCM}
		\label{fig:tcm_gm_new}
	\end{subfigure}
	\caption{The graphical model of state-of-the-art TCM and simplified TCM.}
\end{figure}
