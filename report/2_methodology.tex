\section{Methodology}
\label{sec:methodology}
We compare UBM, DCM, DBN and TCM on the Yandex dataset \cite{yandex}. In this section, we briefly describe their main characteristics and differences. We have implemented TCM ourselves, the other algorithms were taken from PyClick \cite{PyClick}.

\subsection{DCM}
The Dependent Click Model (DCM) was first proposed by Guo et al. in \cite{Guo2009}. In the paper they propose a new click model which can handle multiple clicks per query by introducing a position dependent parameter $lambda_j$ to reflect the chance that the user would like to see more results after a click at position $j$. A graphical representation of the model is presented in Figure\ref{fig:dcm_gm} 

\begin{figure}[ht!]
  \begin{center}
  	\begin{tabular}{c}
  		\input{graph/model_dcm}
  	\end{tabular}
  \end{center}
  \caption{The graphical model of DCM.}	
  \label{fig:dcm_gm}
\end{figure}

\subsection{DBN}
The Dynamic Bayesian Network is an extension to the traditional Cascade model proposed by Chapelle and Zhang in \cite{Zhang2011}. For a given position $j$, in addition to observed variable $C_j$ indicating whether there was a click or not at this position, the following latent variable are defined to model examination, perceived relevance and actual relevance, respectedly:
\begin{itemize}
	\item $E_j$: did the user $examine$ the document?
	\item $A_j$: was the user $attracted$ by the document?
	\item $S_j$: was the user $satisfied$ by the clicked document?
\end{itemize}
They introduce a variable $s_u$ for each document $u$ which describes the relevance of the document for this query. When the user clicks on this document, there is a certain chance that the user will be satisfied. If the user is not satisfied, he continues to examine the next document with a probability $\gamma$ and stops otherwise. The parameter $\gamma$ is known as the 'perseverance'. A graphical representation of the model is presented in Figure \ref{fig:dbn_gm}. 

\begin{figure}[ht!]
	\begin{center}
		\begin{tabular}{c}
			\input{graph/model_dbn}
		\end{tabular}
	\end{center}
	\caption{The graphical model of DBN.}	
	\label{fig:dbn_gm}
\end{figure}

\subsection{UBM}
In \cite{Dupret2008}, Dupret and Piwowarski propose a new click model called the User Browsing Model (UBM). The main difference between UBM and other models is that UBM takes the distance into account from the current document \(u_j\) to the last clicked document \(u_{j'}\) for determining the probability that the user continues browsing:
\[P(E_j =1 \mid C_{j'}=1, C_{j'+1}=0, \dots, C_{j-1}=0) = \gamma_{jj'}\]
The probability that a document at rank \(r\) is examined \(E_j\) therefore depends on all possible paths the user could have taken to arrive at this document:
\[P(E_j = 1) = \sum_{i=1}^{j-1} \gamma_{ji}\]
A graphical representation of the model is presented in Figure~\ref{fig:ubm_gm}.

\subsection{TCM}
% TCM
\begin{figure}[ht]
  \begin{center}
    \begin{tabular}{c}
      \begin{tikzpicture}

  % Define nodes
  \node[obs]                      (n) {$N_i$};
  \node[latent, left=1.2cm of n]  (m) {$M_i$};
  \node[obs, below=0.6cm of n]    (c) {$C_i$};
  \node[latent, left=1.2cm of c]  (e) {$E_{i,j}$};
  \node[latent, right=1.2cm of c] (r) {$R_{i,j}$};
  \node[latent, below=0.6cm of c] (f) {$F_{i,j}$};
  \node[latent, left=1.2cm of f]  (h) {$H_{i,j}$};
  \node[latent, left=1.2cm of h]  (h_prime) {$H_{i,j}'$};
  \node[latent, left=1.2cm of h, yshift=1.2cm]  (e_prime) {$E_{i,j}'$};

  % Connect the nodes
  \edge {m} {n,c} ; %
  \edge {e,r,f} {c} ; %
  \edge {h} {c} ; %
  \edge {h_prime,e_prime} {h} ; %

  % Plates
  \plate [inner sep=.3cm] {p_n} {(e)(c)(r)(f)(h)} {$j=1,\dots,n$} ;
  \plate [inner sep=.3cm] {p_m} {(m)(n)(p_n)} {$i=1,\dots,m$} ;

\end{tikzpicture}
    \end{tabular}
  \end{center}
  \caption{The graphical model of TCM.}
  \label{fig:tcm_gm}
\end{figure}
